\chapter{Camada de Aplicação}

\section{Princípios das aplicações de rede}

Em toda aplicação de rede há sempre um \emph{cliente}, que envia a primeira mensagem, que seria uma solicitação.
O \emph{servidor} então responde à solicitação e há uma troca de mensagens entre eles.

Visto esta forma como se dão as aplicações de rede tem-se uma arquitetura mais empregada, que é a \emph{cliente-servidor}.
A principal característica do servidor é estar sempre on-line, aguardando as solicitações que sempre são iniciadas por clientes, que não necessariamente estão sempre on-line.

Na outra arquitetura possível, tem-se que os hosts da rede tem códigos idênticos, podendo agir como clientes e servidores, fornecendo e solicitando serviços.
Esta arquitetura é a \emph{peer-to-peer} pois os hosts são chamados de \emph{peers}.

As suas vantagens (tanto que elas estão sendo cada vez mais icentivadas) são a escalabilidade natural e 
o menor custo para manter toda uma infra-estrutura de servidores, fazendo com que ela seja considerada \emph{cost effective}.

O que limitação seu emprego são três fatores: a forma como as redes de acesso operam, sendo normalmente assimétricas as bandas up ou down;
o icentivo que tem que ser dado aos hosts (que já não são de uma empresas, pagos por elas e exclusivo para aquele uso) para que eles disponibilizem serviços; e a questão da segurança que advém de não se ter algo centralizado e tudo distribuído, sendo mais difícil saber em quem confiar.

Tendo estas arquiteturas, aborda-se como elas são implementadas.
As aplicações empregam \emph{sockets} que são estruturas mantidas pelos sistemas operacionais responsáveis por abstrair as especificidades da troca de mensagens entre os processos.

Estes sockets são, então, interfaces da camda de transporte, que fornecem alguns serviços para aplicações que a empreguem.
Dentre os serviços desejáveis tem-se a \emph{entrega confiável} (oferecida pelo TCP), que nem sempre deve ser necessária, 
tendo-se aplicações que podem lidar com perda de mensagens sem problemas (e usam UDP, no caso).
Outros serviços estão ligados ao \emph{throughput} e \emph{timing} que poderiam ser garantidos, o que não ocorre na Internet (por se usar comutação por pacotes, em prática).
Eles seriam interessante para uma série de aplicações, que devem lidar com as consequências de perdas da taxa fim-a-fim de transmissão (por uma série de motivos) e variações consideráveis no atraso de entregas.
Uma última garantia é a de segurança, fazendo com que só os aplicativos dos hosts comunicantes tenham conhecimento do conteúdo das mensagens.
Isto é implementado com a ``subcamada'' SSL, que encripta as mensagens antes dela ser enviada para o protocolo de transporte.

Na prática a Internet provê o TCP e o UDP. 
O primeiro é orientado a conexões, o que faz com que uma conexão tenha que ser mantida para haver a troca de dados, 
e tem-se garantias de entrega de mensagens, que são checadas por erros e são entregues em ordem.
O segundo não é orientado a conexão, ou seja, não se mantém estado e as mensagens são ``simplesmente'' enviadas.
A TCP tem também um controle de congestionamento, que permite que as mensagens não inundem a rede e faz com que se tenha certa ``justiça'' na distribuição dos recursos. 
O protocolo UDP ``permite'' que a rede seja inundada por um processo, o que pode ser interessante se quer-se desempenho e a possibilidade de usar todos os recursos necessários.

Outro serviço que é oferecido às aplicações na Internet é o de endereçamento.
Uma aplicação passa a ser unicamente identificada através do \emph{endereço IP} de seu hospedeiro e, como podem haver várias aplicações num mesmo host, a elas também estão associado um \emph{número de porta}.

\section{Protocolos de Aplicação}

Eles são padrões para a comunicação entre um par de processos, cada um agindo como cliente ou servidor em cada conexão.
Define-se uma semântica das mensagens, como se inicia, encerra as conexões e os campos onde serão explicitados o contexto da transmissão de dados.
Além disto, define-se o comportamento, ou seja, quando uma das partes envia uma mensagem e o seu conteúdo.

\subsection{HTTP}

É um dos protocolos mais importantes da Internet, provendo acesso ao conteúdo.
Há duas partes envolvidas: o \emph{browser} cliente e o servidor Web.

O \emph{HyperText Transfer Protocol} possui uma série de \emph{request} do cliente e \emph{response} do servidor.
O servidor, por sua vez, não mantém estado a não ser que se usem \emph{cookies} que são números mantidos no cliente que o servidor pode empregar para localizar dados da conexão (ou de anteriores) em algum banco de dados dele.

As conexões HTTP são de dois tipos: \emph{não-persistentes}, quando uma conexão traz um único objeto ou \emph{persistente} caso múltiplos objetos são enviados pela mesma conexão. 
Dentro das persistentes é possível ainda haver \emph{pipeline}, fazendo com que o cliente solicite um objeto assim que o identificar e sem aguardar o envio do objeto subsequente.
O padrão HTTP 1.0 usa conexões não persistentes, enquanto o 1.1 usa persistentes com pipeline, por padrão.

Quando várias máquinas acessam o mesmo conteúdo pode ser interessante configurar \emph{proxyes} que retornam os objetos imediatamente, caso não seja explicitamente solicitada a versão mais nova, do servidor.

Quanto uma mesma máquina acessa várias vezes o mesmo conteúdo ela mantém o último obtido em cache local.
É possível fazer um \emph{conditional GET} que retorna o objeto somente se ele tiver sido modificado (caso contrário retorna 304).

\subsection{FTP}

Protocolo de transferência de arquivos, sendo usada a porta 21 por padrão.
Esta é a porta da \emph{conexão de controle}, por onde ocorre a autenticação, a navegação e a solicitação de dados.
Os dados per si podem ser enviados em outra conexão, fazendo com que o controle seja \emph{out of band}.

\subsection{E-mail}

Outro serviço importante da Internet, formado por agentes usuários, servidores de e-mail e o protocolo \emph{Simple Mail Transfer Protocol}.
O SMTP é usado entre os servidores para enviar mensagens de um usuário hospedado nele para a caixa de outro usuário em outro servidor.

Entre o usuário e o servidor são usados alguns protocolos. 
Para recebimento das mensagens da caixa tem-se o \emph{POP} e o \emph{IMAP} e pode-se usar alguma interface Web, via HTTP.
Para envio, usa-se o SMTP.

\subsection{DNS}

O \emph{Domain Name Server} é o serviço responsável pela tradução de nomes para endereços IP na Internet.
Ele é um banco de dados distribuído e é um protocolo de aplicação (fica na borda) que roda em UDP.

Ele contém uma hierarquia onde tem-se os servidores raiz (que são poucos pelo mundo) 
que tem endereços de uma série de servidores top-level-domain (.br, .org), 
que tem os endereços dos servidores de instituições, que tem endereços dos hosts.

Na consulta os hosts se conectam a um servidor local. Ele pode ter a resposta em cache ou pode iniciar a consulta.
Há duas formas de fazer a consulta: \emph{recursiva} ou \emph{iterativa}.

Na recursiva o servidor local solicita ao raiz o endereços, este solicita ao próximo responsável e por aí vai até chegar ao autoritativo (o que responde pelo endereço).
Daí o caminho de volta é feito e a resposta chega ao local, que repassa ao cliente.

Na iterativa, o servidor local solicita à raiz o endereço do próximo servidor e ele mesmo o consulta, assim em diante, até obter o endereço final, que é repassado ao cliente.

Os registros do DNS são dos tipo: 
\emph{A} associa nome a IP, 
\emph{CNAME} associa apelido a nome canônico, 
\emph{NS} associa domínio ao seu DNS autoritativo 
e \emph{MX} associa o domínio ao seu servidor de e-mails.

\section{Aplicações P2P}

Há várias formas de arquitetura P2P.

A de diretórios centralizados possui um servidor (como o Napster original) que contém o conteúdo e o endereço dos peers e faz autenticação.
Este é uma espécie de modelo misto, onde há um servidor que centraliza informações importantes.

É possível ter-se um diretório descentralizado, formado por uma série de nós especiais que agem como tal e ao qual estão ligados uma série de nós ``normais''. 
Este modelo é descentralizado e hierárquico.

Outra possibilidade é a forma totalmente descentralizada, onde conhece-se um peer e através dele tenta-se chegar ao destino, inundando a rede com a solicitação, cada um enviando a solicitação a todos os vizinhos. Um exemplo seria o Gnutella.

\subsection{Distribuição de arquivos P2P}

A vantagem desta abordagem aparece quando um arquivo grande deve ser distribuído para uma série de vários hosts distintos.
No modelo cliente-servidor o tempo total de distribuição dependerá dos links de download dos clientes (do mais lento) ou da taxa de upload do servidor,
supondo que as redes de acesso sejam o funil da rede.

Já no modelo P2P os cliente pode obter o arquivo do servidor e distribuí-los com outros nós da rede.
Assim o tempo máximo de distribuição será o que pior for entre as taxas de download mínimo ou upload do server para baixar uma cópia
e distribuição das $N$ cópias com o upload de todos os peers.

A diferença será a escalabilidade, pois o custo do modelo clássico aumenta linearmente com o número de clientes enquanto o P2P tende a não passar de um máximo, independentemente do número de peerrs.

Uma aplicação famosa nesta área é o Bit Torrent. 
Cada conjunto de dados é um torrent que contém um tracker que é um servidor.
O tracker passa aos clientes uma lista de alguns peers do mesmo torrent e com eles são trocados chunks (pedaços do total).
Dentre um peer e seus vizinhos tem-se um esforço de obter os chunks menos frequentes e de fornecer chunks aos vizinhos que enviam dados a maiores taxas.
Isto converge pela escolha ``otimistas'' de vizinhos arbitrários para enviar dados, fazendo com que eles também troquem dados efetivamente.

\subsection{Distributed Hash Table}

Esta aplicação permite gerar um banco de dados distribuídos.
Os dados são localizados nos peers que são mais próximos ao hash do conteúdo buscado.

A fim de facilitar a busca, cria-se uma rede overlay circular (com conceito de sucessor e antecessor) com ``atalhos'' para diminuir o custo de busca e inserção.

\subsection{Comunicação - Skype}

Uma aplicação importante de P2P é o Skype.
Ele possui uma arquitetura P2P hierarquizada, contendo supernós (em uma rede overlay) ao qual estão associados uma série de nós.

O banco de usuários on-line é obtido destes supernós que também tem o papel de intermediar a comunicação P2P entre peers que estiverem atrás de NAT.

