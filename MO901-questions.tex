\chapter{Questões}

\section{Capítulo 1}

\subsection{Q4}
Quais são os dois tipos de serviços de transporte que a Internet provê às suas aplicações? Cite algumas características de cada um destes serviços

Um é baseado em conexão e outro não. 
O primeiro provê entrega confiável, ordenada, controle de fluxo e de congestionamento. 
A outra não provê nenhuma garantia de entrega, mas, por outro lado, não limita o taxa de envio nem tem o overhead de conexão e informações.

\subsection{Q5}
Afirma-se que controle de fluxo e controle de congestionamento são equivalentes. Isso é válido para o serviço orientado para conexão da Internet? Os objetivos do controle de fluxo e do controle de congestionamento são os mesmos?

Não. O controle de fluxo visa evitar que um emissor envie dados a uma taxa maior que o destino pode aceitar e processar.
O de congestionamento visa evitar que a rede fique sobrecarregada de dados, inviabilizando a troca de dados e disperdiçando os recursos da rede.

\paragraph{Q7}
Qual é a vantagem de uma rede de comutação de circuitos em relação a uma de comutação de pacotes? Quais são as vantagens da TDM sobre a FDM em uma rede de comutação de circuitos?

As vantagens das redes comutadas por circuito são as garantias dadas à transmissão de dados. 
Como antes da transmissão são alocados recursos nos nós intermediários, tem-se uma taxa garantida de transmissão e atrasos controlados.
Assim, a priori, toda a transmissão se dá sem jitter, com atraso constante e a taxas pré-estabelecidas.

A vantagem do TDM é fornecer a conexão a totalidade da taxa de transmissão restrita a um intervalo, enquando na FDM tem-se uma taxa limitada durante o tempo todo.

\paragraph{Q9}
Suponha que exista exatamente um comutador de pacotes entre um computador de origem e um de destino. As taxas de transmissão entre a máquina de origem e a máquina de destino são $R_1$ e $R_2$, respectivamente. Admitindo que um roteador use comutação de apcotes do tipo armazena-e-reenvia, qual é o atraso total fim-a-fim para enviar um pacote de comprimento $L$? (Desconsidere a formação de fila, atraso de propagação e atraso de processamento.)

O pacote é enviado a uma taxa $R_1$ pelo primeiro enlace. Ele então, é armazenado no comutador, o que leva $L/R_1$ segundos.
Passado este tempo, o pacote é enviado pelo segundo link, chegando por completo ao destino em $L/R_2$ segundos.
Assim, o atraso total é de $L(1/R_1 + 1/R_2)$.

\paragraph{Q11}
Suponha que você esteja desenvolvendo o padrão para um novo tipo de rede de comutação de pacotes e precisa decidir se seu rede usará CVs ou roteamento de datagramas. Quais são os prós e os contras da utilização de CVs?

No roteamento por datagramas, cada pacote deve conter o endereço de destino, que sendo único, é mais extenso que somente o CV, que é o identificador da conexão.
Assim o tempo de processamento a cada nó tende a ser menor empregando-se CVs, assim como o tamanho dos cabeçalhos dos pacotes.

Por outro lado, antes de enviar o pacote será necessária uma operação anterior que faça a descoberta da rota entre emitente e destinatário e que preencha as tabelas de cada comutador intermediário.
Neste caso, estes sinais para tal transação terão que conter, de alguma forma, o endereço único do emitente e receptor.
Assim, mesmo empregando CVs haverá um processamento inicial que depende de informações de roteamento.
Se a transmissão for restrita a poucos pacotes, este esforço e atraso iniciais podem passar a ser relevantes, tornando esta abordagem desvantajosa.

\paragraph{Q12}
Cite seis tecnologias de acesso. Classifique cada uma delas nas categorias acesso residencial, acesso coorporativo ou acesso móvel.

Há duas tecnologias que empregam a infra-estrutura telefônica existente, sendo prioritariamente empregadas em acesso residencial: a DSL e a dial-up.
Outra opção para residências é a internet a cabo, que aproveita a infra-estrutura existente para as televisões por assinatura.

O acesso corporativo, pode empregar estes métodos também, mas o custo/benefício seria questionável.
Normalmente eles preferirão links dedicados de comunicação.
Apesar de poder ser usada também para acesso residencial, redes de acesso empregando fibras ópticas são uma alternativa para coorporações.

As principais tecnologias de acesso móvel são as WLANs e as redes celulares.
As primeiras são redes sem fio de curto alcance, contendo uma antena fixa que se conecta à rede cabeda, que normalmente emprega outra tecnologia.
As redes celulares empregam a infra-estrutura para telefonia móvel, ou seja, as antenas que dão cobertura à células de grande extensão.
O mesmo sinal empregado para a comunicação e troca de mensagens de texto pode ser empregada para transporte de dados, ou seja, para acesso à Web.
Apesar de oferecer taxas ainda reduzidas de transmissão, a tecnologia 3G e possíveis novidades tendem a aumentar a largura de banda desta opção.

Dentro de coorporações a tecnlogia predominante é a Ethernet. Ela é uma arquitetura escalável e que permite um acesso em altas velocidades a recursos locais (dentro da mesma rede) e a roteadores que comunicam a rede com o restante da Internet.


\paragraph{Q18}
Modens discados, HFC e ADSL são usados para acesso residencial. Para cada uma dessas tecnologias de acesso, cite uma faixa de taxas de transmissão e comente se a largura de banda é compartilhada ou dedicada.

O acesso dial-up é o que oferece menor velocidade de acesso, sendo limitado a 56kbps, que á taxa máxima que pode ser transferia nos 4Khz de largura de banda da conexão telefônica (o mais que suficiente para voz, porém).
As conexões são dedicadas e empregam todo o canal, impedindo que o telefone seja empregado simultaneamente.

A ADSL emprega também as linhas telefônicas, mas ao contrário da dial-up, emprega uma faixa de frequência maior, tendo canais para voz (os mesmos 4Khz para o telefone), upstream e downstream, sendo superior a taxa de downstream.
As limitações de velocidade da técnica estão diretamente à distância das casas para as centrais telefônicas regionais. 
Com esta tecnologia consegue-se velocidades entre algumas centenas de kbps até 3,6,8 Mbps na maioria das localidades e a largura de banda é dedicada (pois usa a mesma infra-estrutura telefônica, exclusiva para cada residência.

Vale lembrar que em ambos casos as várias conexões são multiplexadas nas centrais telefônicas, mas espera-se que a banda prometida seja mantida nesta operação.

O acesso HFC final é feito através de cabos coaxiais, que ligam várias casas à uma central regional. Pelas características da tecnologia, o canal é compartilhado, assim como a banda.
Chegando a tais estações, o tráfego é enviado via fibra ótica às centrais (ou a outras regionais de hierarquia superior), realizando a multiplexação dos sinais dos diversos cabos. Novamente, espera-se que esta operação não cause perdas de velocidade.
OS acessos nesta mobilidade oferecem atualmente taxas bem variadas, desde algumas centenas de kbps até 6, 8 ou 10 Mbps.

\paragraph{Q19}
Considere o envio de um pacote de uma máquina de origem a uma de destino por uma rota fixa. Relacione os componentes do atraso que formam o atraso fim-a-fim. Quais deles são constantes e quais deles são variáveis?

A rota é formada por uma série de nós intermediários e associa-se o atraso a cada nó. O primeiro atraso é associado à tecnologia store-and-forward empregada. Ele é inversamente proporcional a taxa de transmissão do link, que no exemplo será constante.
O segundo atraso tem é o processamento dos nós, que inclui a verificação da consistência dos pacotes, a extração do cabeçalho e a computação do link de saída a ser empregado. Este atraso é constante e nele costuma pesar mais o cálculo do link de saída.
Outro atraso é o de propagação, que depende da velocidade da luz no meio empregado para transmissão, ou seja, dos links, sendo constante no exemplo.

O único atraso variável é o de enfileiramento. 
Este é o tempo que o pacote permanece retido nos nós aguardando a liberação do link de saída que deve ser empregado. Este tempo é o restante para que o envio do pacote que ocupa o link seja encerrado mais o tempo empregado par enviar todos os pacotes que se encontrem enfileirados assim como o em questão e que tenham maior prioridade de uso do link (normalmente por terem chegando antes).

\paragraph{Q21}
Quais são as cinco camadas da pilha do protocolo da Internet? Quais as principais responsabilidades de cada uma dessas camadas?

A camada superior é a de aplicação. Ela contém protocolos padrões da Internet, empregados pelas aplicações tanto para interagir com os serviços disponibilizados na Internet (como cliente) como para serví-los (como servidor).

Tem-se depois a camada de transporte, que presta serviço à de aplicação, abstraindo fatores associados à transmissão de dados.
Ela contém prioritariamente dois protocolos, o TCP e o UDP. Assim como estes, os protocolos desta camada diferem nas garantias que são dadas à camada superior quanto a transmissão de dados.
Dentre elas tem-se a entrega garantida, entrega em ordem, consistência dos dados.
Além destas garantias, a camada de transporte é responsável por permitir que várias aplicações possam usar simultaneamente a rede, como se a estivessem usando exclusivamente.

A próxima camada é a de rede, que é responsável por abstrair, para a camada de transporte todos os passos intermediários que um pacote deve percorrer antes de chegar ao destino.
Enquanto as camadas acima lidam com transmissões fim-a-fim entre hosts, esta camada é responsável por realizar o cálculos de rotas e realizar a comutação de pacotes tanto nos hosts como nos nós intermediários.
Na Internet este é o papel do protocolo IP, que associa um endereço único a cada host na Internet e permite que os pacotes sejam enviados pelos links corretos, sendo comutados para o próximo link na rota calculada a cada passo, chegando ao destino final.
Nota-se que nesta camada não tem-se a preocupação semântica sobre os dados: não sabe-se de que aplicação eles vem e não se preocupa-se com possíveis perdas, duplicatas, desordenações etc.

Abaixo dela tem-se a camada de enlace. Ela é responsável por realizar a transmissão de dados entre cada par de nós da rede, tendo protocolos específicos para uso do meio, detecção e possível correção de erros. Desta forma, ao contrário das outras, os protocolos desta camada são específicos para a(s) tecnologia(s) de rede empregadas por cada equipamento da rede.

Finalmente tem-se a camada física, que é responsável pelo efetivo envio de bits entre dois equipamentos (nós ou hosts) ligados por um link de conexão.
Assim como a de enlace, esta camada é implementada nos dispositivos de rede empregados, que se ligam diretamente ao link de transmissão.

\paragraph{Questão 23}
Que camadas da pilha do protocolo da Internet um roteador implementa? Que camadas um comutador de camada de enlace implementa? Que camadas um sistema final implementa?

Um roteador implementa todas as camadas que não são fim-a-fim, ou seja, a de rede, enlace e física.

Um comutador de enlace não realiza comutação de pacotes nem cálculo de rotas, empregando apenas as camadas de enlace e rede.

Os hosts implementam todas as camadas.

\subsection{Problema 6}

Este problema elementar começa a explorar atrasos de propagação e de transmissão, dois conceitos centrais em redes de computadores. Considere dois computadores, A e B, conectados por um único enlace de taxa $R$ bps. Suponha que esses computadores estejam separados por $m$ metros e que a velocidade de propagação ao longo do enlace seja de $s$ metros/segundo. O computador A tem de enviar um pacote de $L$ bits ao computador B.

\paragraph{a.} Expresso o atraso de propagação $d_{prop}$ em termos de $m$ e $s$.

$d_{prop} = m/s$ segundos.

\paragraph{b.} Determine o tempo de transmissão do pacote $d_{trans}$, em termos de $L$ e $R$

$d_{trans} = L/R$ em segundos, pois $L$ é em bist e $R$ em bits/segundos.

\paragraph{c.} Ignorando os atrasos de processamento e de fila, obtenha uma expressão para o atraso fim-a-fim

$d = L/R + m/s$, ou seja, a soma dos dois anteriores.

\paragraph{d.} Suponha que o computador A comece a transmitir o pacote no instante $t = 0$. No instante $t = d_{trans}$, onde estará o último bit do pacote?

O primeiro bit do pacote chega em $d = d_{prop}$ a B. O último bit estará no link de transmissão se $d_{prop} * R \geq L$. Caso contrário, estará ainda em algum buffer do host A.

\paragraph{e.} Suponha que $d_{prop}$ seja maior que $d_{trans}$. Onde estará o primeiro bit do pacote no instante $t = d_{trans}$?

Atravessando o link.

\paragraph{f.} Suponha que $d_{prop}$ seja menor que $d_{trans}$. Onde estará o primeiro bit do pacote no instante $t = d_{trans}$?

No buffer do host B.

\paragraph{g.} Suponha $s = 2,5 . 10^8$, $L = 100$ bits e $R = 28$ kbps. Encontre a distância $m$ de forma que $d_{prop}$ seja igual a $d_{trans}$.

$m/s = L/R \implies m = s*L/R = 25.10^9 / 28.10^3 < 10^6 m$

\subsection{Problema 14}
Suponha que dois computadores, A e B, estejam separados a uma distância de 10 mil quilômetros e conectados por um enlace direto de $R = 1$ Mbps. Suponha que a velocidade de propagação pelo enlace seja de $2.5 . 10^8$ metros por segundo.

\paragraph{a.} Calcule o produto largura de banda-atraso $R . t_{prop}$

$10^6 . 10. 10^6 / 2.5 . 10^8 = 4 . 10^4$

\paragraph{b.} Considere o envio de um arquivo de 400 mil bits do computador A para o computador B. Suponha que o arquivo seja enviado continuamente, como se fosse uma única grande mensagem. Qual é o número máximo de bits que está no enlace a qualquer dado instante?

Seja um bit qualquer, ele demora $4 . 10^{-2}$ segundos para chegar a B. Neste tempo, a quantidade de bits que entram no link é proporcional a sua taxa de transmissão.
Assim, há $10^6 * 4 . 10^{-2}$ no canal, ou seja, $4 . 10^4$ bits, que é o produto banda-atraso.

\letra{c} Interprete o produto largura de banda-atraso.

Feito.

\letra{d} Qual é o comprimento (em metros) de um bit no enlace? É maior do que a de um campo de futebol?

O máximo de bits que podem estar simultaneamente no enlace é $4. 10^4$. Supondo ``comprimento uniforme'', ele será de $10 . 10^6 / 4.10^4 = 2,5.10^2 = 250m$. São mais de dois campos de futebol de $100m$.

\letra{e} Derive uma expressão geral para o comprimento de um bit em termos da velocidade de propagação $s$, da velocidade de transmissão $R$ e do comprimento do enlace $m$.

$c = m/(t_{prop} * R) = m(m/s * R) = s/R$.

\subsection{Problema 16}
Considere o problema 14, mas agora com um enlace de $R = 1$ Gbps.
\letra{a} Calcule o produto largura de banda-atraso, $R . t_{prop}$

$10^9 * 10.10^6 / 2,5.10^8 = 10^9 * 4.10^{-2} = 4.10^7$ bits.

\letra{b} Considere o envio de um arquivo de 400 mil bits no computador A para o computador B. Suponha que o arquivo seja enviado continuamente, como se fosse uma única grande mensagem. Qual será o número máximo de bits que estará no enlace a qualquer dado instante?

Como $4.10^7 > 400.10^3 = 4.10^5$, todo arquivo estará no enlace.

\letra{c} Qual é o comprimento (em metros) de um bit no enlace?

$ c= s/R = 2,5.10^8/10^9 = 2,5.10^{-1} = 25cm$

\subsection{Problema 17}
Novamente com referência ao problema 14.
\letra{a} Quanto tempo demora para enviar o arquivo, admitindo que ele seja enviado continuamente?

$L/R = 400.10^3/10^6 = 400.10^{-3} = 0,4 = t_{trans}$ e $t_{prop} = 10.10^6 / 2,5.10^8 = 4.10^{-2} = 0,04$.
O atraso será $t_{trans} + t_{prop} = 0,44$ segundos.

\letra{b} Suponha agora que o arquivo seja fragmentado em dez pacotes e que cada pacote contenha 40 mil bits. Suponha que cada pacote seja verificado pelo receptor e que o tempo de transmissão de uma verificação de pacote seja desprezível. Finalmente, admita que o emisso não possa enviar um pacote até que o anterior tenha sido reconhecido. Quanto tempo demorará para enviar o arquivo?

Assim serão 10 pacotes de 40 mil bits. Cada pacote tem atraso de $L/R + m/s = 40.10^3/10^6 + 10.10^6/2,5.10^8 = 40.10^{-3} + 4.10^{-2} = 80.10^{-3}$.
Além disto, cada confirmação demora $40.10^{-2}$ para chegar, levando a um atraso total de $10*(120.10^{-3}) = 1,2$ segundos.

\letra{c} Compare os resultados de `a' e `b'.

A transmissão por fluxo contínuo é muito mais rápida.

\subsection{Problema 20}
Em redes modernas de comutação de pacotes, a máquina de origem segmenta mensagens longas de camada de aplicação (por exemplo, uma imagem ou um arquivo de música) em pacotes menores e os envia pela rede. A máquina destinatária, então, monta novamente os pacotes restaurando a mensagem original. Denominamos esse processo \emph{segmentação de mensagem}. A Figura 1.21 ilustra o transporte fim-a-fim de uma mensagem e sem segmentação. Considere que uma mensagem de $7,5 . 10^6$ bits de comprimento tenha de ser enviada da origem ao destino da Figura 1.32. Suponha que a velocidade de cada enlace seja 1,5 Mbps. Ignore atrasos de propagação, de fila e de processamento.
\letra{a} Considero o envio da mensagem da origem ao destino \emph{sem} segmentação. Quanto tempo essa mensagem levará para ir da máquina de origem até o primeiro comutador de pacotes? Tendo em mente que cada comutador usa comutação de pacotes do tipo armazena-e-reenvia, qual é o tempo total para levar a mensagem da máquina de origem à máquina de destino?

O tempo até o primeiro comutador é $t = 7,5. 10^6 / 1,5 . 10^6 = 5s$. Para chegar ao destino, o tempo é o triplo pois deve-se passar por outro comutador e chegar ao destino final.

\letra{b} Agora suponha que a mensagem seja segmentada em 5 mil pacotes, cada um com 1.500 bits de comprimento. Quanto tempo demorará para o primeiro pacote ir da máquina de origem até o primeiro comutador? Quando o primeiro pacote está sendo enviado do primeiro ao segundo comutador, o segundo pacote está sendo enviado da máquian de origem ao primeiro comutador. Em que instante o segundo pacote terá sido completamente recebido no primeiro computador?

O tempo até o primeiro comutador é $t = 1,5.10^3/1,5.10^6 = 10^{-3}s$. O segundo pacote é completamente recebido no segundo comutador no instante $t' = 2.10^{-3}$.

\letra{c} Quanto tempo demorará para movimentar o arquivo da máquina de origem até a máquina de destino quando é usada segmentação de mensagem? Compare este resultado com sua resposta na parte `a' e comente.

O primeiro pacote demora $3.10^{-3}s$ para chegar à origem. Após ele chegam a cada $10^{-3}s$ um pacote. 
Assim o tempo total de transmissão é $3 . 10^{-3} + 4999 . 10^{-3} = 5002 . 10^{-3}s$.

\letra{d} Discuta as desvantagens da segmentação de mensagem.

No modelo de comutadores tipo store-and-forward é interessante ter mensagens pequenas, pois se mantém menos dados nos comutadores, eles precisam menos espaço e podem lidar com um fluxo mais heterogêneo de mensagens. Tem-se um atraso maior usando-se segmentação, apesar de ser bastante pequeno no exemplo.
Além disto, com os cabeçalhos, a quantidade de dados a serem transmitidos com segmentação é maior.

\section{Capítulo 2}

\paragraph{Q6} Que informação é usada por um processo que está rodando em um hospedeiro para identificar um processo que está rodando em outro hospedeiro?

É necessário saber o endereço do host do processo e o número da porta associada ao processo.

\paragraph{Q9} O que significa protocolo de apresentação?

É uma troca de mensagens entre dois hosts a fim de estabelecer uma conexão entre eles e iniciar uma troca de dados.

\paragraph{Q12} Qual a diferença entre o HTTP persistente com paralelismo e HTTP persistente sem paralelismo? Qual dos dos é usado pelo HTTP/1.1?

O usado é o com paralelismo. Nele o cliente pode requisitar mais objetos antes de receber o primeiro objeto requisitado. No outro caso, cada solicitação e envio de objetos são sequenciais.

\paragraph{Q15} Por que se diz que o FTP envia informações de controle 'fora da banda'?

Pois cria uma nova conexão para transferência de dados, não usando aquela empregada para navegação e controle.

\paragraph{Q16} Suponha que Alice envie uma mensagem a Bob por meio de uma conta de e-mail da Web (como o Hotmail), e que Bob acesse seu e-mail por seu servidor de correio usando POP3. Descreva como a mensagem vai do hospedeirao de Alice até o hospedeiro de Bob. Não se esqueça de relacionar a série de protocolos de camada de aplicação usados para movimentar a mensagem entre os dois hospedeiros.

Via HTTP ela vai para o servidor, que repassa via SMTP para o outro servidor, que é acessado via POP pelo destinatário.

\paragraph{Q19} É possível que o servidor Web e o servidor de correio de uma organização tenham exatamente o mesmo apelido para um nome de hospedeiro (por exemplo, \texttt{foo.com}? Qual seria o tipo de RR que contém o nome do hospedeiro do servidor de correio?

Sim, é possível. O servidor de correio associado ao domínio é armazenado em entradas do tipo MX.

\paragraph{Q22} O servidor UDP descrito na seção 2.8 precisava de uma porta apenas, ao passo que o servidor TCP descrito na seção 2.7 precisava de duas portas. Por quê? Se o servidor TCP tivesse de suporta $n$ conexões simultâneas, cada uma de um hospedeiro cliente diferente, de quantas portas precisaria?

O servidor TCP possui duas portas: uma recebe as solicitações de conexão e cria outra porta por onde o tráfego de dados se dá efetivamente.
Para suportar $n$ conexões são necessárias $n+1$ portas.

\paragraph{Q23} Para a aplicação cliente-servidor por TCP descrita na seção 2.7, por que o programa servidor deve ser executado antes do programa cliente? Para aplicação cliente-servidor por UDP descrita na seção 2.8, por que o programa cliente pode ser executado antes do programa servidor?

Porque a conexão tem que ser estabelecida e, para tal, o servidor deve estar ativo. Caso contrário os dados do cliente não chegarão a serem enviados.
No UDP se o cliente é executado antes do servidor não haverá nenhuma saída, mas o programa funciona pois não é necessária uma conexão prévia.

\paragraph{P6}
%TODO

\paragraph{P7}
%TODO

\paragraph{P9}
%TODO

\paragraph{P23}
%TODO

\section{Capítulo 3}

\paragraph{Q1} Fonte é $y$ e o destino é $x$.

\paragraph{Q2} Esta escolha depende da aplicação. 
Se ela tem mecanismos para lidar com (ou não muda nada para ela) a perda e o desordenamento de pacotes, é interessante. 
Se ela quer velocidade e usar completamente a banda, ou seja enviar mensagens a qualquer hora e a qualquer taxa e ter pacotes com cabeçalhos menores (e assim, transmissão mais rápida).
Além disto se a rede entre os hosts é rápida e com muitos poucos erros.

\paragraph{Q3} Sim, é possível. 
Será necessário implementar os mecanismos de entrega confiável (possivelmente em ordem) na camada de aplicação.
Para tal é necessário identificar a perda de pacotes (com ACKs e timers ou através de sequência semântica da aplicação) e numerá-los para que a entrega seja em ordem.

\paragraph{Q4} a. falso (alguns ACKs são obrigatórios)
b. falso (o controle de congestionamento e de fluxo o fazem)
c. verdade (é o princípio do controle de fluxo)
d. falso (o número depende do tamanho do pacote)
e. verdade
f. falso (depende do do DevRTT, pois o SampleRTT tem peso 1/8)
g. falso (os números de ACK e sequência são independentes no mesmo segmento)

\paragraph{Q5} a. tem X - 90 bytes de dados
b. 90, pois o ACK é dado para o último número de sequência recebido em ordem

\paragraph{Q6} Enviará um seg com seq=44 e ack=80 e receberá duas respostas com seq=82 e ack=...

\paragraph{Q7} Metade, ou seja, $R/2$.

\paragraph{Q8} Falso. Quando o timeout do emitente expira o tamanho da janela cai para 1 MSS e o threshold vai para a metade do tamanho da janela quando houve o timeout.

\paragraph{P10} O protocolo do bit alternante pode falhar se houver reordenamento de mensagens.

\paragraph{P11} No segundo caso seria interessante, pois o envio poderia ser imediato (sem aguardo de confirmações) e somente as raras falhas seriam reportadas e os segmentos equivalentes reenviados. Neste caso a recepção poderia ser fora de ordem e seria necessário reordená-la.
No primeiro caso não seria interessante, pois as perdas não seriam conhecidas pois há poucos dados enviados e o gap pode não ser reconhecidos (ou vai demorar).

\paragraph{P12} Tem-se que o $\frac{nL/R}{RTT + L/R}$ é o uso da rede. Como tem-se $0.027\%$ precisa-se $n = 90/0.027 = 3333.3333 < 3334$.

\paragraph{P19a} Quando recebe-se um ACK para um determinado número de sequência a base é incrementada e este valor fica fora da janela. Se o número de sequência máximo não for suficiente, pode-se receber uma ACK ``perdido'' e que já foi reenviado muito tempo depois. Neste caso, ele será descartado.

\paragraph{P19b} Pelo mesmo motivo.
\paragraph{P19c e d} Verdade. Em ambos os casos se envia uma mensagem e se descartam outras mensagens fora de ordem até que um ACK volte.

\paragraph{P20} a. o número de sequência é associado ao tamanho. Assim, com $2^{32}$ valores é possível mandar este número de bytes.

b. O atraso de transmissão será de $n.L/R$. O número de segmentos é de $2^{32}/1460 = 2941758$. O tamanho de cada é $1460 + 66 = 1526$ e o tempo de transmissão será de $2941759 . 1526 .8 / 10^6$.

\paragraph{P25} Pois a entrega é fora de ordem. Assim, supondo o ACK para o segmento $x$ é bem possível que o segmento $x+1$ seja recebido após o $x + 2$ e por aí vai.

\paragraph{P27} Entre as rodadas 1 e 6 e 23 e 26 teve-se a duplicação da janela a cada rodada (RTT). Está-se na partida lenta.
Entre as rodadas 6 e 15, entre 17 e 22 está-se no controle de congestionamento, visto fato que a janela cresce em 1 MSS.
Na rodada 15 houveram 3 ACKs duplicados e a janela caiu pela metade.
Na rodada 22 houve um timeout estourado.
O valor do threshold inicial era de 32 MSS, que é o valor da janela que causou o incío do congestion avoidance. 
Na rodada 16 ele caiu para 21, pois houve detecção de congestionamento e ele era 42 antes (na rodada 15).
Na rodada 22 a janela era 26 e caiu para 13 após a detecção do congestionamento.
Rodadas e segmentos: 1,1; 2,3; 3,7; 4, 15; 5, 31; 6, 63; 7, 96. Assim, o segmento 70 foi enviado na rodada 7.
Na rodada 26 a janela tem tamanho 8 e threshold de 13. Perdendo-se um pacote nesta rodada teria-se janela e threshold de tamanho 4.

\paragraph{P30} O controle de fluxo TCP limita o tamanho da janela do emissor a partir do espaço livre no buffer do receptor. Como este é suficiente para manter todo o fluxo este limite não existiria.
Mesmo sendo a taxa de chegada de dados ao TCP do emitente superior à capacidade do canal em 10 vezes, o controle de congestionamento não vai reduzir o tamanho da janela, pois não há perdas ou duplicação de ACKs.
Como o buffer de saída é $1\%$ do tamanho do arquivo, se ele tiver tamanho menor que $R$ bits ele também passará a ser uma limitação.
Assim, a janela máxima possível (que seria de $R$ bytes) nunca seria atingível.

\paragraph{P33} A vantagem é ter uma janela inicial bastante grande e poder enviar muitos pacotes logo no primeiro timeout. A questão são as outras conexões. Se a rede tiver tomado um estado de grande tráfego, o envio de uma quantidade imensa de pacotes irá aumentar o problema. Até porque vários deles não irão chegar. Neste caso, empregando o Reno terá uma perda e a janela cairá para apenas um pacote.

\subsection{Problema 34}

Nela faz-se o cálculo da latência para envio de dados continuamente passados pela aplicação para o outro host usando uma conexão TCP.
São dados o $RTT$, o tamanho $O$ do arquivo, o MSS $S$, o tamanho $W$ da janela de transmissão e taxa de transmissão $R$.

A conexão demanda uma troca de mensagens e se solicita os dados no segundo envio. Então o primeiro bit de dados chegam em $2RTT$.
Após isto, o primeiro ACK é enviado após $S/R$ que é o tempo de chegada do primeiro MSS.

Se este ACK chegar ao emissor antes do final do envio dos $W$ segmentos da janela, o envio será contínuo.
Assim, se $RTT + S/R < W(S/R)$ então o tempo total de envio será de $2.RTT + O/R$.

Caso contrário, o emissor ficará aguardando pelo ACK sem enviar, ou seja, por $S/R + RTT - W(S/R)$.
Assim, seja $K = O/(WS)$ a quantidade de janelas que são necessárias para o envio de todos os $O$ bits de dados.
Este tempo de inatividade do emissor ocorrerá $K-1$ vezes.
Assim, o tempo de envio será $2.RTT + O/R + (K-1).(S/R + RTT - W(S/R))$.

Em resumo, o tempo total para receber um fluxo de dados é a soma: 
(i) do tempo até receber o primeiro bit (depende do RTT); 
(ii) o tempo para a transmissão de todos os dados (depende da taxa e do tamanho do arquivo);
(iii) o tempo de ociosidade do emissor entre as $K$ janelas que serão enviadas, que depende do tamanho da janela, do tamanho do MSS, da taxa e do RTT.

Os valores dados são o $RTT = 100ms$, $O = 100kbytes$ e $S = 536 bytes$.
Pergunta-se a latência mínima e o $W$ mínimo necessário, com taxas $R_1 = 28kbps$, $R_2 = 100kbps$, $R_3 = 1Mbps$ e $R_4 = 10 Mbps$.

É feito para $R_1 = 28kbps$ e para os outros é equivalente.
A latência mínima ocorre se a janela for grande a ponto para que $RTT + S/R \le W(S/R)$.
Neste caso a latência é dada por $2.RTT + O/R = 2.0.1 + 100.1024.8/28.1000 = 29.2$ segundos.
Para tal $W \ge (RTT + S/R)/(S/R) = 1.65$, então $W = 2$.
\section{Capítulo 4}

\paragraph{Q2}

%TODO

\paragraph{Q8}

%TODO

\paragraph{P7}

%TODO

%Questões: 19, 21, 22, 24, 32, 35, 36
%Problemas: 21, 23
