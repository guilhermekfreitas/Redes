\chapter{Camada de Rede}

Esta é uma das camadas mais importantes e mais desafiadoras da Internet.
O seu papel é realizar a entrega de pacotes entre dois hosts quaisquer, 
através do \emph{roteamento} e \emph{forwarding} dos pacotes nos nós intermediários da rede.
Assim, esta é uma das camadas presentes no core da rede.

O \emph{forwarding} é a operação realizada pelos \emph{packet switches}, recebendo pacotes em uma interface (enviadas por um nó através de um enlace) 
e enviando-os por outra interface (outro enlace, levando-os a outro nó da rede).
Esta decisão é feita através de uma busca em uma \emph{forwarding table} pela interface de entrada + bits obtidos do cabeçalho do pacote.
Se tal cabeçalho é o da camada de rede tem-se um \emph{roteador}.

Já o \emph{roteamento} é um cálculo de um caminho entre dois hosts quaisquer na rede, ou seja, a definição de uma série de nós intermediários que devem ser passados por um pacote para alcançar o destino.
Em cada um destes nós, ocorre o \emph{forwarding} segundo uma tabela que é atualizada através de um protocolo de roteamento.

Em alguns modelos de rede (que não incluem o modelo da Internet) esta camada também é responsável por realizar a \emph{conection setup}.
Ela consiste na reserva de recursos nos nós intermediários da rede antes que a comunicação propriamente dita (a troca de dados) ocorra.

\section{Modelos de serviço}

É possível que a camada de rede forneça \emph{garantias de entrega}: pacotes cheguem e possivelmente dentro de um período de tempo definido.
Além disto, é possível que se tenha \emph{entrega ordenada} e \emph{garantias de segurança} como encriptação, integridade de dados e autenticação da origem.
É possível estabelecer outras formas mais flexíveis de qualidade de transmissão, como \emph{jitter máximo} e \emph{taxa de transmissão mínima}.

A Internet usa um modelo de \emph{best-effort} que não garante nada disto.
Nas redes \emph{ATM} tem-se várias garantias, divididades em duas classes:

\paragraph{Constant Bit Rate} Interessante para tráfego telefônico e multimídia. Há reserva de recursos e de uma taxa fixa de transmissão. 
Há limites aceitáveis de jitter e de perda ou atraso.
Antes das transmissões as conexões devem ser iniciadas e tais valores são negociados com a rede ATM.
Não há congestionamento.

\paragraph{Avaliable Bit Rate} Modo mais flexível, onde se garante uma taxa mínima de transmissão 
e caso haja recursos disponíveis, a transmissão pode usar maior banda.
A entrega é ordenada, podem haver perdas e o controle de congestionamento é realizado pela camada de rede, com células especiais de controle.

\subsection{Circuitos Virtuais}

É o modelo de transporte na camada de rede que oferece uma conexão entre dois hosts na rede.
Assim, antes do envio de dados há uma etapa de inicialização.
Dado o endereço do destinatário calcula-se o caminho (uma série de enlaces e roteadores) que são informados da conexão através de mensagens de sinalização.

Cada roteador recebe uma mensagem do tipo a partir de um enlace $a$ e com um identificador $CV_i$ os associa a um outro enlace $b$ e outro $CV_{i+1}$.
Esta quádrula $a, CV_i, b, CV_{i+1}$ é armazenada na tabela de forwarding e mantida lá até que receba-se outra mensagem que encerre tal conexão.
Assim, há uma transmissão entre os dois hosts interessados (ida e volta) que permite a configuração da rota (através de cada forward)
e uma possível negociação e alocação dos recursos necessários.

Durante o envio dos dados o mesmo caminho é sempre empregado. 
Cada nó do percurso preenche o campo $CV$ especificado na tabela e o coloca na interface indicada, 
sendo que os nós intermediários adicionalmente leem o cabeçalho do pacote para realizar a busca na tabela.

\subsection{Comutação de Pacotes}

Este é o modelo empregado pela Internet.
Para enviar uma mensagem um host preenche um endereço (único na rede) no cabeçalho do pacote e o põem na rede.
Não há o conceito de conexão, ou seja, o caminho não é previamente conhecido e nenhum recurso da rede é alocado.

Cada roteador extrai o endereço e busca na tabela pela interface para onde ele deve redirecionado.
Como os endereços podem ser grandes, emprega-se uma tabela de prefixos e procura-se pela \emph{longest prefix matching rule}.

Nota-se que nenhum estado é mantido nos roteadores relacionado a determinada comunicação.
Esta tabela, porém, é modificada por algoritmos de roteamento, mas com muito menos frequência que a tabela dos CVs.
Isto também faz com que pacotes de um mesmo fluxo de dados tomem destinos diferentes, não havendo garantia de ordenação na entrega.

\paragraph{}

A distinção deste modelos advém de suas origens.
Os CVs se originam na telefonia, cuja característica é ter-se um núcleo da rede complexo e uma borda ``burra''.
Já na Internet tem-se interesse que o núcleo da rede seja simples (a ponto de possibilitar que várias tecnologias sejam usadas) e a complexidade se encontra na borda.

\section{Funcionamento dos roteadores}

Os roteadores são nós da rede que tem função de fazer o \emph{forwarding} de pacotes, que passam pelos seguintes módulos:
\begin{itemize}
	\item Portas de entrada: realizam as funções das camadas de enlace e física, obtendo um pacote de camada de rede, do qual se extrai o cabeçalho
	\item Processador de roteamento: mantém as tabelas de \emph{forwarding}, participando de protocolos de roteamento e de funções de manutenção da rede
	\item \emph{Switching fabric}: envia o pacote para a porta de saída indicada na tabela de \emph{forwarding}
	\item Portas de saída: recebem um pacote que é encapsulado em um frame da camada de enlace e enviado para o próximo nó. Neles normalmente se encontram as filas e nelas que ocorre a maior parte dos descartes de pacotes
\end{itemize}

Os roteadores antigos (ou atuais para redes de menor velocidade) funcionam de forma similar a um computador,
tendo interfaces de entrada e de saída ``comuns'' e um barramento interno com memória e um processador. 
Um pacote que chegasse era copiado para a memória, analisado pelo processador e copiado para outra região de memória, associada a uma porta de saída.
Este modelo é chamado de \emph{switching} via memória, e é fácil notar que somente um pacote pode ser redirecionado por vez.

Roteadores mais modernos já possuem um processador associado a cada interface, responsáveis por processar pacotes provenientes do seu enlace,
com base em cópias locais da tabela de \emph{forwarding}.
A função de roteamento e atualização das tabelas é realizada por uma outra unidade, separando as tarefas de maneira mais eficiente.

A questão então é como enviar os pacotes da porta de chegada para a de saída calculada por um processador.
Uma abordagem é via \emph{bus}, onde um barramento é compartilhado por todas as portas e passa a ser o maior gargalo para a velocidade do equipamento.
É possível também usar múltiplos barramentos internos, gerando uma ``rede'' dentro do roteador com vários caminhos entre portas e inclusive certos protocolos internos e bem simples de roteamento.

Em todos estes casos é interessante notar que as consultas à tabela de roteamento devem ser realizadas muito rapidamente.
A limitação dos processadores e de métodos eficientes de consulta foram, por grande tempo, um empecilho para a comutação por pacotes, 
que possui endereços maiores que os labels dos circuitos virtuais.

A outra questão importante é a análise de como ocorre o enfileiramento de pacotes.
Se o tráfego de chegada for maior que a capacidade de processamento, consulta às tabelas e repasse à porta destino, pacotes podem ficar enfileirados nas portas de entrada, que precisam de alguma forma de buffer.
Se a limitação for a capacidade dos enlaces de saída, usar os mesmos buffers pode levar a um fenômeno de \emph{head-of-line (HOL) block}.
Ele ocorre quando pacotes cujas portas de destino estão livres ficam enfileirados pois um pacote à sua frente na fila de entrada está bloqueado pela indisponibilidade de sua porta de saída.
Assim, o enfileiramento por excesso de tráfego normalmente se dá nas portas de saída.

É importante notar que é neste enfileiramento que se implementam as políticas de qualidade de serviço e o controle de congestionamento, caso existam.
Assim, são necessários algoritmos para decidir (a) quais pacotes serão descartados quando não houver mais espaço em buffer (e, possivelmente, sinalizar o seu descarte, o congestionamento) 
e (b) quais pacotes da fila serão transmitidos quando o enlace de saída estiver livre (ou seja, que fluxos terão maior prioridade).

\section{Protocolo IP}

É resposável pelo endereçamento, formato e manipulação dos datagramas na Internet.
Há duas versões atuais: o \emph{IPv4} mais utilizado e o \emph{IPv6} como uma nova proposta, já empregada em alguns nichos.

O IP transporta os dados dentro de datagramas, cujo cabeçalho tem as informações necessárias para seu devido encaminhamento ao destino.
Os primeiros campos são de versão do IP empregada (4 bits), o ``tipo de serviço'' (não muito empregado, com 8 bits) o tamanho do datagrama (16 bits) e o tamanho do cabeçalho (4 bits).
Este tem tamanho mínimo de 20 bytes podendo ser extendido com palavras de 4 bytes com parâmetros extras.

Dois campos importantes são o endereço de partida e o de destino, com 32 bits cada.
Há também o campo \emph{TTL} ou tempo de vida, que é o número de hops que o pacote ainda pode percorrer (8 bits), um identificador da camada superior para onde o pacote deve ser repassado (8 bits) e um \emph{checksum do cabeçalho}, em complemento de dois em 16 bits.
Este checksum é checado por cada roteador e recalculado, visto que o campo TTL é alterado. Isto é um pequeno gargalo.

Outro conjunto de campos são relacionados à \emph{fragmentação} do pacote, que ocorre quando ele tem que passar por um enlace cujo frame é menor que o datagrama.
Sabendo desta possibilidade, o emissor assinala um identificador de 16 bits ao pacote. 
Ao fragmentá-lo, o roteador quebra o payload em valores múltiplos de 8 bytes cujo início é marcado no campo \emph{fragment offset} de 13 bits.
Todos os fragmentos a menos do último tem um bit de \emph{flag} setado para 1.
A remontagem do pacote só se dá no destino usando o \emph{id} do pacote, os \emph{offsets} e bits.
Sendo remontado, ele é passo ao protocolo de destino, sendo normalmente descartando se um fragmento não chega.
Esta política de fragmentação foi removida do \emph{IPv6} por conta do custo para os roteadores e a vulnerabilidade desta estratégia (falsos offsets, eternos fragmentos).

\subsection{Endereçamento IPv4}

Os endereços contém 32 bits, levando a $2^{32}$ (cerca de 4 bilhões) de endereços, apesar de várias faixas serem reservadas.
Apesar dele ser um endereçador de hosts, cada \emph{interface} de algum equipamento de rede possui um endereço IPv4 próprio.
Ele é formado por um prefixo de \emph{subnet} de bits mais significativos e os demais bits endereçam interfaces dentro de uma mesma subrede.
Dentre estes bits menos significativos é possível definir outras subredes com outros prefixos, hierarquicamente.

Assim, além de endereços IPv4 únicos tem-se \emph{endereços de redes} formados por um prefixo de $x$ bits e com os demais bits nulos.
Estes $x$ bits são a máscara da (sub-)rede, que pode comportar $2^{32-x} - 2$ endereços IPv4 distintos.
Com esta estrutura é possível \emph{agregar} endereços e reduzir o tamanho das tabelas de roteamento, 
com um roteador divulgando o endereço da maior subrede ``abaixo'' dele.
É possível, porém, que alguns endereços desta faixa ``abaixo'' dele não estejam sobre seu controle, 
sendo divulgadas por outro roteador que as possuem com uma máscara $y < x$.
Como o associação para o redirecionamento se dá pelo \emph{maior} prefixo, esta estratégia funciona.

Esta forma de organização é chamada de \emph{Classes Interdomain Routing} ou \emph{CIDR}.
Inicialmente tinha-se classes \emph{A,B,C} com máscaras \emph{8,16,24}, mas dado a imensa diferença de número de IPv4 associadas acabaram sendo flexibilizadas para máscaras de tamanhos arbitrários.

Para finalizar, a unicidade destes endereços é regulamentada por uma organização chamada \emph{ICANN}, que é descentralizada entre outras organizações regionais.
Ela é responsável por alocar faixas de IPs e também pela alocação de nomes DNS, neste caso na resolução de conflitos de nomes.

\subsubsection{DHCP}

O \emph{Domain Host Configuration Protocol} tem como principal objetivo automatizar a configuração de um endereço IP único dentro de uma subnet para os hosts.
Além do endereço IP normalmente se configura a máscara de rede, o primeiro roteador (ou \emph{gateway}), o endereço do DNS local, dentre várias outras possibilidades.
Um parâmetro importante é o \emph{lease-time}, que é a validade do endereço, o que permite ao servidor rehaver um endereço que foi fornecido e não é mais usado (e cujo host não ``devolveu'' antes de sair).
Nota-se que com isto é possível atender não simultaneamente mais hosts do que os endereços disponíveis.

Assim, um host ingressante irá enviar uma mensagem de \emph{discover} para a rede no endereço \emph{255.255.255.255} de broadcast e para a porta 67 UDP.
A resposta será esperada no endereço \emph{0.0.0.0} (\emph{myself}) na porta 68 UDP e tem-se um ID de transação.
Um servidor DHCP que receba a mensagem irá responder com um \emph{offer} com o mesmo ID em broadcast, adicionando o endereço IP sugerido e demais informações.
O cliente que o receber responderá com um \emph{request} em broadcast, incrementando o ID incrementado e adicionando o endereço do servidor escolhido.
Este responde em broadcast com um \emph{ack} e é finalizado o processo.
Nota-se que tudo é feito em broadcast, pois podem haver vários servidores DHCP na rede: o cliente é livre de escolher o que mais lhe convir, 
mas os demais também ter ciência desta escolha (caso contrário, podem pensar que alguma mensagem foi perdida!).

\subsubsection{NAT}

A quantidade e a forma com que os IPv4 foram distribuídos podem fazer com que eles sejam insuficientes em muitos casos para todos os hosts de uma rede.
Além disto, há faixas do endereçamento IPv4 que são reservadas para ``redes privadas''.
Elas claramente não são identificadores únicos na Internet, pois hosts em subredes distintas podem ter o mesmo IP,
das faixas \emph{10.0.0.0/8}, \emph{172.16.0.0/12} e \emph{192.168.0.0/16}.

Assim, uma solução para prover acesso à Internet para vários computadores de uma rede do tipo é o \emph{Network Address Translation}.
Nela o roteador tem um IPv4 dito ``real'' em uma interface e uma rede privada em outra. 
Toda esta rede é vista na Internet como um único endereço: o do roteador.

O papel do roteador é fazer com que a troca de pacotes ocorra. Como os que estão na rede privada são clientes, eles abrem conexões TCP.
Antes do roteamento, o roteador muda os campos de proveniência do pacote para seu endereço e a porta para uma porta sua qualque não alocada.
Em uma \emph{tabela de tradução NAT} ele mantém o endereço e porta originais do cliente e o endereço e portas dele próprio com o qual ele sobrescreveu os campos do pacote original.
Ao checar uma resposta, ele checa esta tabela, restaura os valores sobrescritos e tudo fica transparente ao cliente.

O problema é quando o host ``atrás'' de um NAT é o servidor. 
Neste caso, como ele não tem endereço único na rede ele é inalcançável: pacotes enviados para ele serão recebidos pelo roteador que não saberá o que fazer deles.
As soluções para este problema são a \emph{connection reversal} onde o servidor contacta o cliente (fora do NAT), 
configuração manual de uma associação ``inversa'' no roteador (com uma tabela de pós roteamento) 
ou o protocolo \emph{UPnP} que negocia com o roteador esta associação de forma dinâmica.

Enfim, o NAT é visto por muitos como um remendo aos problemas do IPv4. 
Além destas limitações do host ficar ``invisível'' na Web, tem-se o limite de conexões que podem usar o NAT (são $2^{16}$ portas disponíveis no cabeçalho TCP) e toda a questão que portas são de camada de transporte para assignar processos, não para assignar hosts, que é uma solução crosslay etc.
Por outro lado, a solução pode ser bem interessantes para redes onde se há dinamismos e somente clientes e não se quer comprar mais endereços para os novos hosts (como residências).
E, finalmente, o NAT hoje é amplamente empregado e difundido.

\subsubsection{Protocolo ICMP}

O \emph{Internet Control Message Protocol} é considerado com parte do protocolo IP, sendo responsável pela sua sinalização de erros.
Por outro lado, os pacotes ICMP são enviados no payload de um datagrama IP, assim como os segmentos TPC e UDP.

Um pacote ICMP tem um campo de tipo e outro de código e contém os primeiros 8 bytes do cabeçalho do pacote IP que gerou o erro.
Ele é empregado para sinalizar erros de construção de datagramas IP, erros encontrados no processamento do pacote no host de destino
(porta inexistente, protocolo inexistente) e impossibilidade de entrega do pacote ao host ou rede destino, por eles serem inexistentes ou inalcançáveis.

Além disto, ele retorna erro quando o TTL do pacote é zerado e tem funções de ping (\emph{echo request} e \emph{echo reply}) e
para traçar e descobrir rotas.
Há também um mecanismo de notificação de congestionamento, que não é empregado.

\subsubsection{IPv6}

A principal razão do seu desenvolvimento é a possibilidade de que em breve não se tenham mais faixas de IPv4 disponíveis.
Assim uma das principais modificações é um novo endereço de 128 bits (mais que suficiente para muito tempo) que possui também possibilidade de endereçamento \emph{anycast}, ou seja, para um grupo de hosts.
Além desta modificação alguns fatores complicados do IPv4 foram eliminados: a fragmentação, o checksum e a possibilidade de opções extras no cabeçalho.

O cabeçalho IPv6 é de tamanho fixo de 40 bytes e o tamanho do pacote é a 40 mais o \emph{payload lenght} (16 bits).
Opções extras podem ainda ser passadas, mas devem ser alocadas no payload e usa-se o campo \emph{next header} de 8 bits para apontar para elas.
Normalmente este campo aponta para o protocolo para qual o payload será repassado (UDP, TCP, ICMP etc.).
No caso de pacotes muito grandes para o enlace, ele são descartados e uma mensagem ICMP (há uma nova versão também deste protocolo) é enviada ao emissor.
Tem-se um campo \emph{hop limit} (que é o uso prático do TTL do IPv4) alterado a cada roteador, mas já sem o recálculo do checksum, 
que foi eliminado pela suposição que o protocolo superior o faz, assim como muitas vezes o frame de enlace de onde o datagrama veio.

Há outras novidades para determinação de \emph{classe de tráfego} com 8 bits e um \emph{flow label} de 16 bits.
Com eles seria possível estabelecer políticas de encaminhamento para pacotes a depender da classe e prioridade para certos fluxos.
A forma como isto seria feito não foi efetivamente determinado.

Um grande problema é como fazer a transição entre estas duas tecnologias. 
Como os cabeçalhos são distintos, eles não podem coexistir sem complicações: mesmo que os roteadores sejam capazes de traduzir os cabeçalhos de uma versão para outra, alguns campos seriam perdidos ou inutilizados.
A opção mais usada atualmente é o \emph{tunelamento} dos datagramas IPv6 no payload de datagramas IPv4 realizadas por roteadores que sabem lidar com as duas versões do protocolo.

\subsubsection{Segurança sobre IP}

Nesta seção, cita-se o \emph{IPsec} que é um protocolo que roda sobre IP e tem opções extra de segurança.
Ele estabelece sessões entre hosts o que permite a encriptação (através de uma chave negociada na conexão) dos payloads dos datagramas,
tem mecanismos extras de integridade de dados e permite a autenticação das origens.
Isto é feito com uma camada acima do protocolo IP, que precisa coexistir em versões idênticas somente nos hosts finais, passando transparente pelos roteadores.


\section{Algoritmos de roteamento}

Esta seção lida com os algoritmos básicos para o cálculo das tabelas de roteamento em uma rede.
Supõe-se que cada host está conectado a alguma rede de acesso e possui um roteador mais vizinho, a partir do qual o tráfego para fora desta subrede sai e por onde ele vem.
Assim, o roteamento entre dois hosts se resume ao cálculo da ``melhor'' rota entre estes um roteador de partida a um de destino.

Esta rede de roteadores é modelada como um grafo onde às arestas são associados custos, cuja significância pode ser bastante variada.
Ele pode estar relacionado ao tempo de propagação, o seu custo econômico, à políticas administrativas etc.
Nesta modelagem, o roteamento vai consistir em obter um caminho entre dois roteadores cujo custos das arestas intermediárias sejam mínimos.

Os métodos empregados podem ser classificados por suas características, por requerem conhecimento global ou local da rede, 
por serem dinâmicos ou estáticos e sendo sensíveis ao tráfego ou não.

\subsection{Algoritmo de Estado de Enlaces}

Este algoritmo calcula os caminhos a partir do conhecimento dos custos de todas as arestas do grafo de comunicação (é \emph{global}).
Cada roteador têm o conhecimento dos custos dos enlaces a eles conectados e os distribui a todos os demais roteadores da rede.
Esta distribuição é feita via broadcast quando se nota uma modificação de algum dos custos e/ou com certa periodicidade.

Com a chegada de novos custos e/ou novas arestas, as distâncias mínimas são computadas a partir do algoritmo de Dijkstra.
Para cada destino $v$ mantém-se o custo mínimo para alcançá-lo e nó $u$ que é o último roteador no caminho até $v$.
O caminho completo pode ser obtido através da consulta dos dados referentes a $u$ e assim recursivamente até que o nó seja algum vizinho do nó de origem.

Um problema desta abordagem advém dela ser sensível ao tráfego, o que pode gerar muita oscilação de rotas.
Isto ocorre quando vários caminhos mínimos compartilham um mesmo enlace, cujo uso excessivo vai fazer com que seu custo aumente.
Este novo custo vai ser divulgado e várias rotas evitarão este enlace, sobrecarregando um outro enlace.
É difícil evitar este problema, que advém de uma natural ``sincronização'' da computação de rotas.

\subsection{Algoritmo de Vetor de distância}

Este outro algoritmo depende somente de \emph{informação local} e não é sensível ao tráfego, além de ser assíncrono e iterativo.
Inicialmente, cada roteador calcula os custos para os vizinhos e os custos para os demais roteadores é tido como infinito.
Estes custos são repassado para os vizinhos, que obtém então o menor custo para nós a dois passos de comunicação, assim como o enlace por onde ele é obtido.
Assim, as informações locais são disseminadas iterativamente pela rede sempre, através do envio para os vizinhos dos novos valores conhecidos sempre que eles são alterados.

Após um certo número de iterações o algoritmo converge e novas atualizações são geradas somente quando há mudança de custos de links.
Se diz que para este algoritmo ``boas notícias'' correm rápido, pois a redução significativa do custo de um link é rapidamente disseminada pela rede.
Por outro lado, se o custo de um link aumenta (ou o link cai) o comportamento é complicado, havendo grandes possibilidades da ocorrência de loops de roteamento.

Um roteador $v$ que nota uma queda ou o aumento do custo de um link para um roteador $u$, 
irá verificar quais dos seus vizinhos divulga um custo mais vantajoso, digamos $w$.
A questão é quando esta rota divulgada por $w$ para $u$ passava por $v$ (e pelo link afetado). 
$v$ irá atualizar seu custo para $u$ e divulgá-lo. $w$ irá notar a atulização e aumentará seu custo para $u$, pois ele passava por $v$.
Isto irá se repetir por tempo indeterminado até que que uma nova rota efetivamente melhor pareça mais atraente.
Enquanto isto não ocorrer, $v$ irá repassar para $w$ os pacotes destinados a $v$ e $w$ irá lhe devolver o mesmo pacote (pois a melhor rota para $u$ passa por $v$) e tem-se um loop.

Este problema é chamado de \emph{contagem até o infinito} e para ser resolvido é necessário que não somente o custo, 
mas também dados sobre a rota que resulta nele sejam disseminados para os vizinhos.

\paragraph{Comparações}

Estes são os dois métodos principais (e quase únicos) de roteamento na Internet.
Uma grande diferença entre eles é que no DV um nó comunica somente com seus vizinhos informações de roteamento para todos nós,
enquando no LS um nó comunica para todos os nós da rede informação dos links a eles adjacentes.
Assim, a complexidade teórica de mensagens para o LS é maior, assim como o custo para computar as melhores rotas, o que aliás tende a ocorrer sempre que houver mudança de custo de um link.
Por outro lado, o DV sofre de problemas de contagem até o infinito (ou seja, convergência lenta) e possíveis loops de roteamento.
Uma informação incorreta ou mal transmitida levará a erros na computação de algumas rotas no LS, 
mas no DV o repasse uma informação incorreta de custos é facilmente propagada pela rede e pode causar problemas graves.

\section{Roteamento na Internet}

Na prática, por questões de escala e autonomia administrativa, os protocolos de roteamento não são executados por todos roteadores da Internet.
Ao invés disto, um grupo de roteadores administrado por uma mesma organização é um \emph{Automnomous system (AS)}.
A Internet então é vista como um conjunto de ASs, interligados entre si, e cujo roteamento de pacotes é realizado de forma hierárquica.

Dentro de um AS emprega-se um protocolo de roteamento \emph{intra-AS}, escolhido pelos seus administradores 
e que define rotas para subredes sob responsabilidade de outros roteadores do mesmo AS.
Este roteamento é feito segundo princípios de eficiência e velocidade de comunicação até porque a topologia intra-AS pode ser conhecida por todos ou ser atualizada rapidamente.

O outro nível de roteamento é o \emph{inter-AS}, que permite a comunicação entre redes associadas a diferentes ASs.
Um AS possui um conjunto de \emph{gateway routers} que são conectados a outros ASs, ditos vizinhos.
Através do protocolo inter-AS tais roteadores divulgam a seus vizinhos 
uma série de prefixos que remetem às subredes sobre seu controle, tornando-as acessíveis na Internet.
Além disto, um AS pode divulgar rotas conhecidas para outros ASs, fazendo com que o trafégo para estes outros ASs passe por eles.

O protocolo inter-AS é único na Internet e seu funcionamento está mais relacionado a acordos comerciais para tráfego de dados do que a métricas de eficiência e velocidade de comunicação.
Assim um AS que possua uma rota para um outro AS pode decidir por não divulgá-la, sendo atravessado por tráfego somente com destino a seus prefixos.
Neste caso eles são conhecidos como \emph{AS stubs}.
Outros ASs realizam também (ou principalmente) a conexão entre redes, sendo conhecidos como \emph{backbones}.
Vale notar aqui que um ISP é formado por um AS ou por mais, e neste caso o intuíto é de eficiência ou melhor organização das rotas.

\subsection{Protocolo Intra-AS: OSPF}

É um protocolo de especificação livre, sendo um dos mais empregados em ASs de ISPs de mais alto nível.
Ele é um protocolo de estado de enlaces, cujos custos são determinados pelos administradores e disseminados pela rede.
O OSPF roda sobre o protocolo IP, tendo módulos para transmissão confiável, teste de links e roteadores vizinhos e
permite a comunicação entre roteadores de forma encriptada e autenticada.
Outras características são a possibilidade de várias rotas entre um mesmo par de hosts e mecanismos específicos para rotear mensagens de broadcast e multicast.

Com o OSPF tem-se a possibilidade de realizar roteamento hierárquico intra-AS, através da divisão do AS em \emph{areas}.
Cada área tem um ou mais \emph{roteadores de borda} que são responsáveis por roteador pacotes para fora da área.
Uma área especial é chamada de \emph{backbone} e conecta os roteadores de bordas, que fazem o roteamento para dentro das suas áreas.
No backbone também estão os gateways do AS e somente por ele que passa o tráfego destinado a outros ASs.

\subsection{Protocolo Inter-AS: BGP}

O \emph{Border Gateway Protocol} é o protocolo de roteamento inter-AS da Internet, tendo como principais papéis:
\begin{itemize}
	\item[i] Obter as redes atingíveis a partir dos ASs vizinhos
	\item[ii] Propagar as redes antigíveis a partir de roteadores dentro do AS
	\item[iii] Determinar rotas ``boas'' para as subredes, baseando-se em quesitos ``políticos'' (normalmente, econômicos)
\end{itemize}

A comunicação entre roteadores se dá por conexões TCP semi-permanentes na porta 179 entre roteadores.
Roteadores de AS diferentes se conectam diretamente através do \emph{eBGP} 
e roteadores de um mesmo AS se conectam (não necessariamente entre vizinhos) através do \emph{iBGP}.

O roteamento inter-AS é realizado através dos identificadores únicos \emph{ASN} de cada AS, designados pela ICANN (que também associa endereços IPs).
Vale notar que AS stubs que não são intermediários em rotas entre outros ASs não possuem tais identificadores.
As mensagens de divulgação de rotas contém vários parâmetros, dos quais se destacam:
\begin{itemize}
	\item O prefixo, resultado da agregação de várias redes acessíveis por um AS
	\item O \emph{AS-PATH} que é a sequência de ASs que formam a rota, empregados no roteamento inter-AS realizado pelo \emph{eBGP}
	\item O \emph{NEXT-HOP} que indica o endereço da interface de saída de um gateway, empregada pelo roteamento intra-AS realizado pelo \emph{iBGP}
\end{itemize}

Ao receber uma divulgação do tipo, um roteador decidirá se irá incorporá-la, através de uma \emph{política de importação}.
O primeiro parâmetro analizado é político/econômico, determinado manualmente, seguido do tamanho do caminho e da distância para o next hop.

\paragraph{}
Assim, em cada roteador agem dois protocolos distintos de roteamento: o OSPF para destinos intra-AS e o BGP para destinos em outros ASs.

Uma rota é divulgada por um AS para seus vizinhos via eBGP, que possivelmente as repassa aos próximos vizinhos.
Via iBGP o gateway repassa as rotas para destinos externos para os nós internos do AS. 
Estes irão preencher suas tabelas tendo como rota para o destino externo a rota que leva ao gateway correspondente, indicado pelo NEXT-HOP da mensagem.
Havendo duas possibilidades distintas de rotas para um mesmo destino emprega-se a estratégia \emph{hot potato}, 
que escolhe como rota preferencial aquela advinda do roteador cujo custo intra-AS de roteamento seja mínimo.

